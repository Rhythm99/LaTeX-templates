%%%%%%%%%%%%%%%%%%%%% Article Template %%%%%%%%%%%%%%%%%%%%%
\documentclass[11pt,a4paper,leqno]{article}
\usepackage[width=15.00cm, height=23.00cm]{geometry}

\usepackage{palatino} % For palatino style, may be used/removed. 
\usepackage{amsmath,amsfonts,amssymb,amsxtra,color,calligra,mathrsfs,comment,url,amsthm}
%\let\circledS\undefined % here - PS
%\usepackage[bitstream-charter]{mathdesign}

\usepackage{authblk}

\usepackage{xcolor}
\colorlet{mdtRed}{red!50!black}
\colorlet{dblue}{blue!50!black}
\usepackage[colorlinks]{hyperref}
\hypersetup{linkcolor=dblue,citecolor=dblue,filecolor=dullmagenta,urlcolor=mdtRed}

\usepackage[all]{xy}
\usepackage{tikz,tikz-cd,tkz-graph,enumerate}
\usetikzlibrary{matrix,arrows,decorations.pathmorphing}

%%%%%%%%%%%%%%% Some shortcuts %%%%%%%%%%%%%%%%%
\DeclareMathOperator{\Hom}{\textnormal{Hom}}
\DeclareMathOperator{\sHom}{\mathcal{H}\!\textit{om}}
\DeclareMathOperator{\sEnd}{\mathcal{E}\!\textit{nd}}
\DeclareMathOperator{\rk}{\mathrm{rk}}
\DeclareMathOperator{\Id}{\textnormal{Id}}
\DeclareMathOperator{\At}{\textnormal{At}}
\DeclareMathOperator{\ad}{\textnormal{ad}}
\DeclareMathOperator{\Aut}{\textnormal{Aut}}
\DeclareMathOperator{\Lie}{\textnormal{Lie}}
\DeclareMathOperator{\GL}{\textnormal{GL}}
\DeclareMathOperator{\Der}{\mathcal{D}\!{\it er}}

%%%%% New commands: 
\newcommand{\mf}[1]{\mathfrak{#1}}
\newcommand{\mc}[1]{\mathcal{#1}}
\newcommand{\scr}[1]{\mathscr{#1}}
\newcommand{\bb}[1]{\mathbb{#1}}
\newcommand{\dv}{\vee\vee}

\makeatletter
\newcommand{\subjclass}[2][2010]{%
	\let\@oldtitle\@title%
	\gdef\@title{\@oldtitle\footnotetext{#1 \emph{Mathematics subject classification.} #2}}%
}
\newcommand{\keywords}[1]{%
	\let\@@oldtitle\@title%
	\gdef\@title{\@@oldtitle\footnotetext{\emph{Key words and phrases.} #1.}}%
}
\makeatother

\numberwithin{equation}{subsection}

\newtheorem{theorem}[equation]{Theorem}
\newtheorem{corollary}[equation]{Corollary}
\newtheorem{lemma}[equation]{Lemma}
\newtheorem{proposition}[equation]{Proposition}
\newtheorem*{theorem-nonumber}{Theorem}

\theoremstyle{definition}
\newtheorem{definition}[equation]{Definition}
\newtheorem{remark}[equation]{Remark}
\newtheorem{example}[equation]{Example}

\newtheorem*{thm-intro}{Theorem}%[section]
\newtheorem*{cor-intro}{Corollary}
\newtheorem*{prop-intro}{Proposition}
\newtheorem*{lem-intro}{Lemma}
\newtheorem*{rem-intro}{Remark}

\title{Annual Progress Report} % Title to appear in the font page. 
\newcommand\shorttitle{Annual Progress Report} % Short title to appear in odd numbered (>1) pages. 

\author[1]{Arjun Paul}
%\affil[1]{\small{
%		Department of Mathematics\\ 
%		Indian Institute of Technology\\
%		Powai, Mumbai 400076, India\\
%		{Email:} \texttt{arjunp@math.iitb.ac.in}.}
%}

%\keywords{Fundamental group; local ring; smooth manifold}
%
%\subjclass{14J60, 53C07, 32L10}

\date{\today}

\begin{document}

\maketitle
\begin{tabular}{l}
\hline 
	{Arjun Paul} \\
	\hspace{.1in} Post Doctoral Fellow,\\
	\hspace{.1in} Department of Mathematics,\\
	\hspace{.1in} Indian Institute of Technology Bombay, \\
	\hspace{.1in} Powai, Mumbai 400076, Maharashtra, India. \\
	\hspace{.1in} Email: {\rm \texttt{arjun.math.tifr@gmail.com}} \\ 
	\hspace{.1in} Date of Joining at IIT Bombay: February 22, 2019. \\
	\hspace{.1in} Employee Id No. 20001477. \\
\hline 
\end{tabular}

	Name: Arjun Paul 
	Designation: Post Doctoral Fellow 
	Department of Mathematics, 
	Indian Institute of Technology Bombay, 
	Powai, Mumbai 400076, India. 
	Email: arjunp@math.iitb.ac.in 
	Date of joining at IITB: February 22, 2019. 
	Id. No. : 20001477. 
	
	Subject: Progress report during the period “February 22, 2019 to February 21, 2020”.
	
	Research interests: I work on an area of Mathematics, broadly known as algebraic geometry. My research interests includes equivariant bundles, logarithmic connections on bundles, moduli of principal bundles, Higgs bundles, opers, fundamental group schemes of algebraic varieties, algebraic stacks etc. 
	
	Research paper accepted for publication: 
	
	1. “System of Hodge bundles and generalized opers on smooth projective varieties” (joint work with Suratno Basu and Arideep Saha), Journal of Geometry and Physics, Vol 145, November 2019, 103484. doi:10.1016/j.geomphys.2019.103484, (arXiv:1903.11347).
	
	Brief description: Let k be an algebraically closed field of any characteristic. 
	Let X be a polarized irreducible smooth projective algebraic variety over k. In this article, 
	we give criterion for semistability and stability of system of Hodge bundles on X. We define 
	notion of generalized opers on X, and prove semistability of the Higgs bundle associated to 
	generalized opers. We also show that existence of partial oper structure on a vector bundle $E$ 
	together with a connection $\nabla$ over X implies semistability of the pair $(E, \nabla)$. 
	
	Research papers submitted to journals for peer review: 
	
	1. “Fundamental Group Schemes of Hilbert Scheme of n Points on a Smooth Projective Surface”, joint work with Ronnie Sebastian, (preprint available at arXiv:1907.04290). 
	
	Brief description: Let k be an algebraically closed field of characteristic $p > 3$. 
	Let $X$ be an irreducible smooth projective surface over $k$. Fix an integer $n \geq 1$ and let 
	$\mathcal{H}{\it ilb}_X^n$ be the Hilbert scheme parameterizing effective $0$-cycles of length $n$ on $X$. 
	The aim of the present article is to find the $S$-fundamental group scheme and Nori's fundamental group scheme 
	of the Hilbert scheme $\mathcal{H}{\it ilb}_X^n$.  
	
	2. “Fundamental Group Schemes of n-fold Symmetric Product of a Smooth Projective Curve”, joint work with Ronnie Sebastian, (preprint available at arXiv:1907.09388).
	
	Brief description: Let $k$ be an algebraically closed field of characteristic $p > 0$. 
	Let $X$ be an irreducible smooth projective curve of genus $g$ over $k$. Fix an integer $n \geq 2$, 
	and let $S^n(X)$ be the $n$-fold symmetric product of $X$. In this article we find the $S$-fundamental 
	group scheme and Nori's fundamental group scheme of $S^n(X)$.  
	
	3. “Criterion for existence of a logarithmic connection on a principal bundle over a smooth complex projective variety”, joint work with Sudarshan Gurjar, (preprint available at arXiv:1912.00598).
	
	Brief description: Let $X$ be a connected smooth complex projective variety of dimension $n \geq 1$. 
	Let $D$ be a simple normal crossing divisor on $X$. Let $G$ be a connected complex Lie group, and 
	$E_G$ a holomorphic principal $G$-bundle on $X$. In this article, we have given a criterion for 
	existence of a logarithmic connections on $E_G$ singular along $D$. 
	
	Academic Visit: 
	
	April 08-12, 2019: Department of Mathematics and Statistics, Indian Institute of Science Education and Research (IISER) Kolkata. 
	
	Invited Seminar Talk:
	
	1. “Equivariant principal bundle on a complex manifold” at the Department of Mathematics and Statistics at IISER Kolkata (April 10, 2019). 
	
	Conference participated:
	
	1. “Moduli of bundles and related structures”, February 10-14, 2020 at the International Centre for Theoretical Sciences (ICTS), Bangalore, India.
	
	
	Ongoing research works and future research projects:
	
	1. Collaborating with Prof. Sudarshan R. Gurjar on some problem related to semistable bundles on higher dimensional varieties. 
	
	2. Collaborating with Prof. Ronnie Sebastian on some problem related to fundamental group schemes of some algebraic varieties. 
	
	3. Collaborating with Prof. Saurav Bhaumik on some problems related to moduli stacks and Higgs bundles. 
	
	Apart from the above mentioned problems, I am looking at some recent works on derived categories of certain geometric objects, and considering some related problems as long term future research projects. 
	
	
	(Arjun Paul) 
\end{document}